\documentclass{article}
\usepackage[utf8]{inputenc}

\usepackage[english,greek]{babel}
\newcommand{\lt}{\latintext}
\newcommand{\gt}{\greektext}

% Set page size and margins
% Replace `letterpaper' with`a4paper' for UK/EU standard size
\usepackage[a4paper,top=2cm,bottom=2cm,left=3cm,right=3cm,marginparwidth=1.75cm]{geometry}

\usepackage{graphicx}

\title{2η Υποχρεωτική Εργασία \\
Αριθμητική Ανάλυση \\
Χειμερινό Εξάμηνο 2021-2022}
\author{Αλεξία Νταντουρή - 3871}
\date{23 Ιανουαρίου 2022}

\begin{document}

\maketitle

\section{Άσκηση 5}
Για να προσεγγίσω την συνάρτηση του ημιτόνου χρησιμοποίησα τις παρακάτω τιμές του ημιτόνου με ακρίβεια 4 δεκαδικών ψηφίων.
\lt
\begin{center}
\begin{tabular}{ |c|c|c|c|c|c|c|c|c|c| }
\hline
x & y\\
\hline
-3 & -0.1411\\
-2.7 & -0.4274\\
-2.1 & -0.8632\\
-1.2 & -0.9320\\
-0.4 & -0.3894\\
0 & 0\\
0.9 & 0.7833\\
1.4 & 0.9854\\
2.5 & 0.5985\\
3.1 & 0.0416 \\
\hline
\end{tabular}
\end{center}
\gt
Παρακάτω, φαίνεται το σφάλμα προσέγγισης για 200 σημεία στο $[-\pi,\pi]$ χρησιμοποιώντας τις μεθόδους:
\begin{itemize}
  \item Πολυωνυμική προσέγγιση \lt (Lagrange) \gt
  \item \lt Splines \gt
  \item Ελάχιστα τετράγωνα
\end{itemize}

\begin{center}
\includegraphics[height=9cm]{interpolation_error.png}
\end{center}
[Αρχείο: \lt \textbf{ask5errors.py}] \gt \\
Φαίνεται πως η πολυωνυμική προσέγγιση έχει τα μικρότερα σφάλματα σε σχέση με τις άλλες 2 μεθόδους. Οι \lt splines \gt έχουν επίσης μικρά σφάλματα, αλλά μεγαλύτερα σε σχέση με την πολυωνυμική προσέγγιση. Τέλος, φαίνεται πως η μέθοδος ελαχίστων τετραγώνων έχει τα μεγαλύτερα σφάλματα σε σχέση με τις άλλες δύο μεθόδους.

\subsection{Πολυωυμική προσέγγιση}
[Αρχείο: \lt \textbf{ask5polynomial.py}] \gt \\
Προγραμμάτισα σε \lt python \gt μία συνάρτηση \lt \textbf{lagrange{\_}interpolation(xp)}, \gt η οποία επιστρέφει την προσέγγιση στο \lt xp \gt με βάση το ποlυώνυμο παρεμβολής του \lt Lagrange. \gt Η συνάρτηση αυτή, καλεί την συνάρτηση \lt \textbf{calc{\_}L(i, xp)}, \gt η οποία επιστρέφει το $L_i(xp)$, το οποίο χρησιμοποιείται στον υπολογισμό του πολυωνύμου \lt Lagrange. \gt \\
Παρακάτω, φαίνεται το σφάλμα προσέγγισης για 200 σημεία στο $[-\pi,\pi]$ χρησιμοποιώντας πολυωνυμική προσέγγιση.
\begin{center}
\includegraphics[height=9cm]{polynomial_error.png}
\end{center}
Φαίνεται, λοιπόν, πως με την πολυωνυμική προσέγγιση πετυχαίνουμε ακρίβεια 3 δεκαδικών ψηφίων.

\lt
\subsection{Splines}
\gt
[Αρχείο: \lt \textbf{ask5splines.py}] \gt \\
Προγραμμάτισα σε \lt python \gt μία συνάρτηση \lt \textbf{splines{\_}interpolation(x0, x, y)}, \gt η οποία δέχεται έναν μονοδιάστατο πίνακα \lt x0, \gt έναν μονοδιάστατο πίνακα \lt x \gt και έναν μονοδιάστατο πίνακα \lt y \gt και επιστρέφει την προσέγγιση \lt f0 \gt  με βάση τα \lt x,y, \gt χρησιμοποιώντας τη μέθοδο των \lt splines. \gt \\
Παρακάτω, φαίνεται το σφάλμα προσέγγισης για 200 σημεία στο $[-\pi,\pi]$ χρησιμοποιώντας \lt splines. \gt
\begin{center}
\includegraphics[height=9cm]{splines_error.png}
\end{center}
Φαίνεται, λοιπόν, πως με τις \lt splines \gt πετυχαίνουμε ακρίβεια 1 δεκαδικού ψηφίου.

\subsection{Μέθοδος ελάχιστων τετραγώνων}
[Αρχείο: \lt \textbf{ask5leastsquares.py}] \gt \\
Προγραμμάτισα σε \lt python \gt μία συνάρτηση \lt \textbf{least{\_}squares(A, b, t)}, \gt η οποία δέχεται έναν δισδιάστατο πίνακα Α, έναν μονοδιάστατο πίνακα \lt b \gt και έναν \lt float t \gt και επιστρέφει την προσέγγιση στο \lt t \gt χρησιμοποιώντας τη μέθοδο ελαχίστων τετραγώνων. Η συνάρτηση αυτή καλεί την συνάρτηση \lt \textbf{form(a, b, c, d, t)}, \gt η οποία ορίζει το παραμετροποιημένο μοντέλο που χρησιμοποιείται για τα δεδομένα (στην προκειμένη περίπτωση πολυώνυμο 3ου βαθμού).\\
Παρακάτω, φαίνεται το σφάλμα προσέγγισης για 200 σημεία στο $[-\pi,\pi]$ χρησιμοποιώντας τη μέθοδο ελαχίστων τετραγώνων.
\begin{center}
\includegraphics[height=9cm]{leastsquares_error.png}
\end{center}
Φαίνεται, λοιπόν, πως με τη μέθοδο ελαχίστων τετραγώνων πετυχαίνουμε ακρίβεια μικρότερη του 1 δεκαδικού ψηφίου.


\section{Άσκηση 6}

\lt
\subsection{Simpson's Rule}
\gt
[Αρχείο: \lt \textbf{ask6SimpsonsRule.py}] \gt \\
Προγραμμάτισα σε \lt python \gt μία συνάρτηση \lt \textbf{simpsons{\_}rule(a, b, n)}, \gt όπου $n$ το πλήθος των υποδιαιρέσεων του διαστήματος $[a,b]$. Η συνάρτηση επιστρέφει την προσεγγιστική τιμή του ολοκληρώματος μιας συνάρτησης $f$ στο διάστημα $[a,b]$ με βάση τον κανόνα του \lt Simpson. \gt Στο πρόγραμμα δημιούργησα τη συνάρτηση \lt $\textbf{f(x)}$, \gt η οποία επιστρέφει την τιμή του ημιτόνου στο \lt x \gt . Επιπλέον, προγραμμάτισα τη συνάρτηση \lt \textbf{error{\_}bound(a, b, n, max)}, \gt η οποία δέχεται τα \lt a, b, n, \gt και το \lt max, \gt όπου $max=|f^{(4)}(x)|: x\in[a,b]$. Η συνάρτηση αυτή επιστρέφει το σφάλμα προσέγγισης του κανόνα \lt Simpson. \gt \\
Στο πρόγραμμα, καλώ την συνάρτηση \lt \textbf{simpsons{\_}rule(a, b, n)}, \gt με $a=0, b=\frac{\pi}{2}$ και $n=10$ και βρίσκω την προσέγγιση 1.000003. \\
Τέλος, καλώ την συνάρτηση \lt \textbf{error{\_}bound(a, b, n, max)}, \gt με $max=1$ και βρίσκω το σφάλμα: $|e|\leq0.000005$ και συγκεκριμένα $e=0.000003$.

\subsection{Μέθοδος τραπεζίου}
[Αρχείο: \lt \textbf{ask6TrapezoidalRule.py}] \gt \\
Προγραμμάτισα σε \lt python \gt μία συνάρτηση \lt \textbf{trapezoidal{\_}rule(a, b, n)}, \gt όπου $n$ το πλήθος των υποδιαιρέσεων του διαστήματος $[a,b]$. Η συνάρτηση επιστρέφει την προσεγγιστική τιμή του ολοκληρώματος μιας συνάρτησης $f$ στο διάστημα $[a,b]$ με βάση τον κανόνα τραπεζίου. Στο πρόγραμμα δημιούργησα τη συνάρτηση \lt $\textbf{f(x)}$, \gt η οποία επιστρέφει την τιμή του ημιτόνου στο \lt x \gt . Επιπλέον, προγραμμάτισα τη συνάρτηση \lt \textbf{error{\_}bound(a, b, n, max)}, \gt η οποία δέχεται τα \lt a, b, n, \gt και το \lt max, \gt όπου $max=|f''(x)|: x\in[a,b]$. Η συνάρτηση αυτή επιστρέφει το σφάλμα προσέγγισης του κανόνα τραπεζίου. \\
Στο πρόγραμμα, καλώ την συνάρτηση \lt \textbf{trapezoidal{\_}rule(a, b, n)}, \gt με $a=0, b=\frac{\pi}{2}$ και $n=10$ και βρίσκω την προσέγγιση 0.99794. \\
Τέλος, καλώ την συνάρτηση \lt \textbf{error{\_}bound(a, b, n, max)}, \gt με \lt max=1 \gt και βρίσκω το σφάλμα: $|e|\leq0.0032$ και συγκεκριμένα $e=0.00206$.


\section{Άσκηση 7}
Επέλεξα να εκτιμήσω την τιμή κλεισίματος των κρυπτονομισμάτων \lt Bitcoin \gt και \lt Ethereum. \gt Για να προσεγγίσω την συνάρτηση τιμής κλεισίματος χρησιμοποιήσα τις τιμές κλεισίματος για 10 συνεχόμενες ημέρες μέχρι και τις 04/04/2021, δηλαδή από τις 26/03/2021 μέχρι και τις 04/04/2021. Έπειτα, προέβλεψα την ισοτιμία των κρυπτονομιστάτων για τις επόμενες 5 ημέρες, δηλαδή από τις 05/04/2021 μέχρι και τις 09/04/2021.
\lt
\begin{center}
\begin{tabular}{ |c|c|c|c| }
\hline
x & Date & BTC & ETH\\
\hline
1 & 26/03/2021 & 55,137.31 & 1,702.84\\
2 & 27/03/2021 & 55,937.51 & 1,716.49\\
3 & 28/03/2021 & 55,950.75 & 1,691.36\\
4 & 29/03/2021 & 57,750.20 & 1,819.68\\
5 & 30/03/2021 & 58,917.69 & 1,846.03\\
6 & 31/03/2021 & 58,918.83 & 1,918.36\\
7 & 01/04/2021 & 59,095.81 & 1,977.28\\
8 & 02/04/2021 & 59,384.31 & 2,143.23\\
9 & 03/04/2021 & 57,603.89 & 2,028.42\\
10 & 04/04/2021 & 58,758.55 & 2,093.12\\
\hline
\textbf{11} & 05/04/2021 & 59,057.88 & 2,107.89\\
\textbf{12} & 06/04/2021 & 58,192.36 & 2,118.38\\
\textbf{13} & 07/04/2021 & 56,048.94 & 1,971.08\\
\textbf{14} & 08/04/2021 & 58,323.95 & 2,088.57\\
\textbf{15} & 09/04/2021 & 58,245.00 & 2,072.11\\
%16 & 10/04/2021 & 59,793.23 & 2,135.94\\
\hline
\end{tabular}
\end{center}
\gt
Προσέγγισα την συνάρτηση τιμής κλεισίματος με πολυώνυμο 2ου, 3ου και 4ου βαθμού με τη μέθοδο ελαχίστων τετραγώνων.

\lt
\subsection{Bitcoin}
\gt
Αρχείο: \lt \textbf{ask7BTC2ndDegPolynomial.py} \gt \\
Αρχείο: \lt \textbf{ask7BTC3rdDegPolynomial.py} \gt \\
Αρχείο: \lt \textbf{ask7BTC4thDegPolynomial.py} \gt \\
Παρακάτω, φαίνονται οι προσεγγίσεις της ισοτιμίας του \lt Bitcoin \gt σε σχέση με τις διαθέσιμες τιμές, καθώς και οι προβλέψεις για τις επόμενες 5 τιμές χρησιμοποιώντας πολυώνυμα 2ου, 3ου και 4ου βαθμού.

\lt
\begin{center}
\begin{tabular}{ |c|c|c|c|c|c| }
\hline
x & Date & BTC & 2nd deg & 3rd deg & 4th deg \\
\hline
1 & 26/03/2021 & 55,137.31 & 54,743.75 & 54,905.97 & 55,303.19\\
2 & 27/03/2021 & 55,937.51 & 55,951.43 & 55,897.35 & 55,411.86\\
3 & 28/03/2021 & 55,950.75 & 56,956.37 & 56,821.19 & 56,446.04\\
4 & 29/03/2021 & 57,750.20 & 57,758.58 & 57,638.85 & 57,705.05\\
5 & 30/03/2021 & 58,917.69 & 58,358.06 & 58,311.71 & 58,708.93\\
6 & 31/03/2021 & 58,918.83 & 58,754.79 & 58,801.14 & 59,198.36\\
7 & 01/04/2021 & 59,095.81 & 58,948.79 & 59,068.53 & 59,134.73\\
8 & 02/04/2021 & 59,384.31 & 58,940.06 & 59,075.24 & 58,700.09\\
9 & 03/04/2021 & 57,603.89 & 58,728.59 & 58,782.66 & 58,297.17\\
10 & 04/04/2021 & 58,758.55 & 58,314.38 & 58,152.16 & 58,549.38\\
\hline
11 & 05/04/2021 & 59,057.88 & 57,697.43 & 57,145.11 & 60,300.81\\
12 & 06/04/2021 & 58,192.36	& 56,877.75 & 55,722.89 & 64,616.23\\
13 & 07/04/2021 & 56,048.94	& 55,855.33 & 53,846.88 & 72,781.10\\
14 & 08/04/2021 & 58,323.95	& 54,630.18 & 51,478.46 & 86,301.52\\
15 & 09/04/2021 & 58,245.00	& 53,202.29 & 48,578.99 & 106,904.32\\
\hline
\end{tabular}
\end{center}
\gt
Παρακάτω, φαίνονται τα σφάλματα των προσεγγίσεων σε σχέση με τις διαθέσιμες τιμές.
\lt
\begin{center}
\begin{tabular}{ |c|c|c|c|c|c| }
\hline
x & Date & BTC & 2nd deg & 3rd deg & 4th deg \\
\hline
11 & 05/04/2021 & 59,057.88 & 1,360.45 & 1,912.77 & 1,242.93\\
12 & 06/04/2021 & 58,192.36	& 1,314.61 & 2,469.47 & 6,423.87\\
13 & 07/04/2021 & 56,048.94	& 193.61 & 2,202.06 & 16,732.16\\
14 & 08/04/2021 & 58,323.95	& 3,693.77 & 6,845.49 & 27,977.57\\
15 & 09/04/2021 & 58,245.00	& 5,042.71 & 9,666.01 & 48,659.32\\
\hline
\end{tabular}
\end{center}
\gt
Παρατηρούμε πως το πολυώνυμο 2ου βαθμού έχει τα μικρότερα σφάλματα, ενώ το πολυώνυμο 4ου βαθμού έχει τα μεγαλύτερα σφάλματα.\\
Παρατηρούμε, επίσης, πως η 1η πρόβλεψη για κάθε πολυώνυμο είναι κοντά στην πραγματική τιμή. Καθώς όμως απομακρυνόμαστε από τις τιμές που χρησιμοποιήσαμε ως δεδομένα, οι προβλέψεις αποκλίνουν όλο και περισσότερο από τις πραγματικές τιμές.\\
Παρακάτω φαίνονται τα διαγράμματα για τα πολυώνυμα 2ου, 3ου και 4ου βαθμού. Τα μπλε σημεία είναι οι τιμές που χρησιμοποιήθηκαν για να γίνει η προσέγγιση και τα πορτοκαλί σημεία είναι οι 5 προβλέψεις με βάση τα αντίστοιχα πολυώνυμα.
\begin{center}
\includegraphics[height=9cm]{BTC2ndDegPolynomial.png}
\end{center}

\begin{center}
\includegraphics[height=9cm]{BTC3rdDegPolynomial.png}
\end{center}

\begin{center}
\includegraphics[height=9cm]{BTC4thDegPolynomial.png}
\end{center}

\lt
\subsection{Ethereum}
\gt
Αρχείο: \lt \textbf{ask7ETH2ndDegPolynomial.py} \gt \\
Αρχείο: \lt \textbf{ask7ETH3rdDegPolynomial.py} \gt \\
Αρχείο: \lt \textbf{ask7ETH4thDegPolynomial.py} \gt \\
Παρακάτω, φαίνονται οι προσεγγίσεις της ισοτιμίας του \lt Ethereum \gt σε σχέση με τις διαθέσιμες τιμές, καθώς και οι προβλέψεις για τις επόμενες 5 τιμές χρησιμοποιώντας πολυώνυμα 2ου, 3ου και 4ου βαθμού.

\lt
\begin{center}
\begin{tabular}{ |c|c|c|c|c|c| }
\hline
x & Date & ETH & 2nd deg & 3rd deg & 4th deg \\
\hline
1 & 26/03/2021 & 1,716.49 & 1,661.05 & 1,707.77 & 1,707.15\\
2 & 27/03/2021 & 1,691.36 & 1,713.10 & 1,697.52 & 1,698.28\\
3 & 28/03/2021 & 1,819.68 & 1,765.01 & 1,726.08 & 1,726.67\\
4 & 29/03/2021 & 1,846.03 & 1,816.80 & 1,782.31 & 1,782.21\\
5 & 30/03/2021 & 1,918.36 & 1,868.45 & 1,855.10 & 1,854.47\\
6 & 31/03/2021 & 1,977.28 & 1,919.96 & 1,933.31 & 1,932.69\\
7 & 01/04/2021 & 2,143.28 & 1,971.35 & 2,005.83 & 2,005.73\\
8 & 02/04/2021 & 2,143.23 & 2,022.60 & 2,061.54 & 2,062.12\\
9 & 03/04/2021 & 2,028.42 & 2,073.73 & 2,089.30 & 2,090.06\\
10 & 04/04/2021 & 2,093.12 & 2,124.71 & 2,077.99 & 2,077.37\\
\hline
11 & 05/04/2021 & 2,107.89 & 2,175.57 & 2,016.50 & 2,011.56\\
12 & 06/04/2021 & 2,118.38 & 2,226.30 & 1,893.69 & 1,879.76\\
13 & 07/04/2021 & 1,971.08 & 2,276.89 & 1,698.45 & 1,668.79\\
14 & 08/04/2021 & 2,088.57 & 2,327.35 & 1,419.64 & 1,365.09\\
15 & 09/04/2021 & 2,072.11 & 2,377.68 & 1,046.15 & 954.79\\
\hline
\end{tabular}
\end{center}
\gt
Παρακάτω, φαίνονται τα σφάλματα των προσεγγίσεων σε σχέση με τις διαθέσιμες τιμές.
\lt
\begin{center}
\begin{tabular}{ |c|c|c|c|c|c| }
\hline
x & Date & ETH & 2nd deg & 3rd deg & 4th deg \\
\hline
11 & 05/04/2021 & 2,107.89 & 67.68 & 91.39 & 96.33\\
12 & 06/04/2021 & 2,118.38 & 107.92 & 224.69 & 238.62\\
13 & 07/04/2021 & 1,971.08 & 305.81 & 272.63 & 302.29\\
14 & 08/04/2021 & 2,088.57 & 238.78 & 668.93 & 723.48\\
15 & 09/04/2021 & 2,072.11 & 305.57 & 1025.96 & 1117.32\\
\hline
\end{tabular}
\end{center}
\gt
Παρατηρούμε πως το πολυώνυμο 2ου βαθμού έχει τα μικρότερα σφάλματα, ενώ το πολυώνυμο 4ου βαθμού έχει τα μεγαλύτερα σφάλματα.\\
Παρατηρούμε, επίσης, πως η 1η πρόβλεψη για κάθε πολυώνυμο είναι κοντά στην πραγματική τιμή. Καθώς όμως απομακρυνόμαστε από τις τιμές που χρησιμοποιήσαμε ως δεδομένα, οι προβλέψεις αποκλίνουν όλο και περισσότερο από τις πραγματικές τιμές.\\
Παρακάτω φαίνονται τα διαγράμματα για τα πολυώνυμα 2ου, 3ου και 4ου βαθμού. Τα μπλε σημεία είναι οι τιμές που χρησιμοποιήθηκαν για να γίνει η προσέγγιση και τα πορτοκαλί σημεία είναι οι 5 προβλέψεις με βάση τα αντίστοιχα πολυώνυμα.
\begin{center}
\includegraphics[height=9cm]{ETH2ndDegPolynomial.png}
\end{center}

\begin{center}
\includegraphics[height=9cm]{ETH3rdDegPolynomial.png}
\end{center}

\begin{center}
\includegraphics[height=9cm]{ETH4thDegPolynomial.png}
\end{center}

\end{document}
